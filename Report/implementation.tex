\section{Implementation}

For each person (author or editor) found in the dblp excerpt, we had to generate
an HTML file with the following path:
\hl{a-tree/first-letter-of-lastname/lastname.firstname.html}.\\

First of all, we have to retrieve each distinct author and editor (there are
multiple publications, so an author/editor can appear multiple times). In order
to do so, the \hl{xsl:for-each-group} element was used, which lets us group
elements and iterate over them. Here, we selected authors and editors, and
grouped them by themselves.\\

\begin{lstlisting}
<xsl:template match="/">
    <xsl:for-each-group select="/dblp/*/author | /dblp/*/editor" group-by=".">
    ...
    </xsl:for-each-group>
</xsl:template>
\end{lstlisting}
\

Then, for each author (or editor), we want to be able to create a file with the
correct name. To do that, we created the \hl{ufn:format\_full\_name()} function
which takes as parameter the full name of a person. The last name is extracted
and put in the first position. Then, if the first name has multiple parts, they
are concatenated with '\_', and if '.' is present, it is then removed (e.g.
\emph{A. B. M. Shawkat Ali} becomes \emph{Ali.A\_B\_M\_Shawkat}).\\

Once the name is correctly formatted, we can use \hl{xsl:result-document} to
produce a document per author/editor.\\

\begin{lstlisting}
<xsl:result-document href="a-tree/{$first_letter}/{$formattedAuthor}.html">
    <html>
        <head>
            <title>Publication of <xsl:value-of select="$author"/></title>
        </head>
        <body>
            <xsl:call-template name="author_header"/>
            <xsl:call-template name="author_publication"/>
            <xsl:call-template name="coauthor_index"/>
        </body>
    </html>
</xsl:result-document>
\end{lstlisting}

\subsection{Publications}
For each person, we store the publications' years in a variable. Then, we iterate
over the publications sorted by year (descending). The variable previously stored let us
retrieve for a publication $p$, the ${p-1}^{th}$ publication's year in order to
know if we must display a year header.\\

\begin{lstlisting}
<xsl:template name="author_publication">
    <xsl:variable name="sortedYears" as="xs:integer*">
        <xsl:perform-sort select="/dblp/*[author = current-grouping-key()]/year |
            /dblp/*[editor = current-grouping-key()]/year">
            <xsl:sort select="." order="descending"/>
        </xsl:perform-sort>
    </xsl:variable>

    ...

    <xsl:for-each select="/dblp/*[author = current-grouping-key()
            or editor = current-grouping-key()]">
        <xsl:sort select="year" order="descending" />

        <xsl:variable name="prevPosition" select="position() - 1"/>
        <xsl:variable name="prevYear" select="$sortedYears[$prevPosition]"/>

        <xsl:if test="not(year = $prevYear)">
            <tr><th colspan="3" bgcolor="#FFFFCC">
                <xsl:value-of select="year"/></th>
            </tr>
        </xsl:if>

        ...
    </xsl:for-each>
\end{lstlisting}
\

\begin{framewarning}
We tested other methods to try to display year headers. The first one used
\hl{previous-sibling} in the \hl{for-each} loop, but it did not work, since the
elements from the loop are sorted by year, and the axis returns elements as if
they were not sorted. The second one used \hl{for-each-group}, grouped by year,
and probably was more readable, less verbose than what is used now. But it was
then hard to retrieve the publications' position, since the \hl{position()}
function returned the position inside the group, and not inside all the author's
publications. Moreover, since XSLT is a functional programming language, variables
are immutable, so implementing a counter was not an option (it can be done
recursively though, but it would not have helped us).
\end{framewarning}
\

Each publication is then handled inside the \emph{publication} named template.
The position is computed with the position in the for each loop and the total
number of publications ($number of publications - position() + 1$). A template
is applied on the publication itself in order to match one of the publication
type (e.g. book, article, masterthesis). Inside of these publication types
templates, templates are applied on sub-elements that can appear in these
publications. By applying templates, optional elements are handled easily.
\newpage

\subsection{Co-authors index}
For the co-author index, publications are sorted by year and stored in a
variable which will be used later.\\

\begin{lstlisting}
<xsl:variable name="sortedPublications">
    <xsl:perform-sort select="/dblp/*[author = current-grouping-key()] |
        /dblp/*[editor = current-grouping-key()]">
        <xsl:sort select="year" order="descending"/>
    </xsl:perform-sort>
</xsl:variable>
\end{lstlisting}
\

Then, coauthors/coeditors are grouped by themselves and iterated over with a
\hl{xsl:for-each-group}. for each distinct coauthor/coeditor, his name is
displayed in the left column of a table, and his coauthored publications
are displayed in the right column. For the coauthored publications, the
\hl{ufn:coauthored\_publications()} function is applied on the current
coauthor/coeditor and the sorted list of publications from the author.\\

\begin{lstlisting}
<xsl:for-each-group select="($sortedPublications/*/author
    | $sortedPublications/*/editor)[not(. = current-grouping-key())]"
    group-by="."
>
    <tr>
        <td align="right"><xsl:apply-templates select="."/></td>
        <td align="left">
            <xsl:sequence select="ufn:coauthored_publications(.,
                $sortedPublications)"/>
        </td>
    </tr>
</xsl:for-each-group>
\end{lstlisting}
\

Inside the \hl{ufn:coauthored\_publications()} function, author's publications
are iterated over, and if the current coauthor/coeditor as participated in this
publication, then the publication number is displayed inside brackets.\\

\begin{lstlisting}
<xsl:function name="ufn:coauthored_publications">
    <xsl:param name="coAuthEd"/>
    <xsl:param name="sortedPublications"/>

    <xsl:for-each select="$sortedPublications/*">
        <xsl:sort select="position()" data-type="number" order="descending"/>

        <xsl:if test="(author | editor)[$coAuthEd = .]">
            [<a href="#p{position()}">
                <xsl:value-of select="position()"/>
            </a>]
        </xsl:if>
    </xsl:for-each>
</xsl:function>
\end{lstlisting}
\newpage
