\documentclass[a4paper, 11pt]{article}

\usepackage[top=2cm, bottom=2cm, left=2cm, right=2cm]{geometry}
\usepackage[utf8]{inputenc}
\usepackage[T1]{fontenc}
\usepackage{indentfirst}

% Multiple columns
\usepackage{multicol}
\setlength{\columnseprule}{1pt} % separation line between columns

% Colors
\usepackage[usenames,dvipsnames]{xcolor}
\definecolor{dkgreen}{rgb}{0,0.6,0}
\definecolor{steelblue}{rgb}{0.16,0.37,0.58}
\definecolor{gray}{rgb}{0.5,0.5,0.5}
\definecolor{mauve}{rgb}{0.58,0,0.82}
\definecolor{blue}{rgb}{0,0,0.7}
\definecolor{hlColor}{rgb}{0.94,0.94,0.94}
\definecolor{shadecolor}{rgb}{0.96,0.96,0.96}
\definecolor{TFFrameColor}{rgb}{0.96,0.96,0.96}
\definecolor{TFTitleColor}{rgb}{0.00,0.00,0.00}
\definecolor{lightred}{rgb}{1,0.96,0.96}
\definecolor{darkred}{rgb}{0.85,0.33,0.31}
\definecolor{lightblue}{HTML}{EBF5FA}
\definecolor{lightblue2}{HTML}{E3F2FA}
\definecolor{darkblue}{HTML}{D2DCE1}
\definecolor{lightyellow}{HTML}{FFFAE6}
\definecolor{darkyellow}{HTML}{FAE6BE}

\usepackage{hyperref}
\hypersetup{
	colorlinks=true,	% false: boxed links; true: colored links
	linkcolor=black,	% color of internal links
	urlcolor=blue,		% color of external links
	citecolor=blue
}

% highlight
\usepackage{soul}
\sethlcolor{hlColor}

% Figures & graphics
\usepackage{graphicx}	% import graphics
\usepackage{wrapfig}	% wrap text around figures
\usepackage{subcaption} % subfigures

% Colored frames
\usepackage{mdframed}
\usepackage{framed}

\newenvironment{framehint}{%
	\begin{mdframed}[backgroundcolor=lightblue, linecolor=darkblue]%
}{\end{mdframed}}

\newenvironment{framehint2}{%
	\begin{mdframed}[backgroundcolor=lightblue2, linecolor=darkblue]%
}{\end{mdframed}}

\newenvironment{framewarning}{%
	\begin{mdframed}[backgroundcolor=lightyellow, linecolor=darkyellow]%
}{\end{mdframed}}

\newenvironment{frameurgent}{%
	\begin{mdframed}[backgroundcolor=lightred, linecolor=darkred]%
}{\end{mdframed}}

% Leftbar
\newlength{\leftbarwidth}
\setlength{\leftbarwidth}{1pt}
\newlength{\leftbarsep}
\setlength{\leftbarsep}{10pt}

\newcommand*{\leftbarcolorcmd}{\color{gray}}

\renewenvironment{leftbar}{%
    \def\FrameCommand{{\leftbarcolorcmd{\vrule width \leftbarwidth\relax\hspace {\leftbarsep}}}}%
    \MakeFramed {\advance \hsize -\width \FrameRestore }%
}{%
    \endMakeFramed
}

% Code listings
\usepackage{listings}
\lstset{
	language=XML,
	basicstyle=\scriptsize,
	numbers=left,                   % where to put the line-numbers
  	numberstyle=\tiny\color{gray},
	commentstyle=\color{steelblue},
	stringstyle=\color{BrickRed},
	backgroundcolor=\color{shadecolor},
    keywordstyle=\color{OliveGreen},
	frame=single,                   % adds a frame around the code
 	rulecolor=\color{black},
	emph={},
	emphstyle=\color{mauve},
	morekeywords=[2]{},
	keywordstyle=[2]{\color{dkgreen}},
	showstringspaces=false,
  	tabsize=4,
	moredelim=[is][\small\ttfamily]{/!}{!/},
	breaklines=true
}

% Title page
\title{
	\textbf{INFO-H-509 - Project2 : XSLT}\\
}
\date{\today}

\begin{document}
\maketitle

\section{Introduction}

The goal of this assignment was to transform an excerpt of the dblp bibliography
into html files for each author and editor using XSLT.

\begin{framehint}
    All the results from the transformation performed with the XSLT stylesheet
    on the dblp's excerpt were retrieved and stored in the \emph{Out} directory.
\end{framehint}

\section{Implementation}

For each person (author or editor) found in the dblp excerpt, we had to generate
an HTML file with the following path:
\hl{a-tree/first-letter-of-lastname/lastname.firstname.html}.\\

First of all, we have to retrieve each distinct author and editor (there are
multiple publications, so an author/editor can appear multiple times). In order
to do so, the \hl{xsl:for-each-group} element was used, which lets us group
elements and iterate over them. Here, we selected authors and editors, and
grouped them by themselves.\\

\begin{lstlisting}
<xsl:template match="/">
    <xsl:for-each-group select="/dblp/*/author | /dblp/*/editor" group-by=".">
    ...
    </xsl:for-each-group>
</xsl:template>
\end{lstlisting}
\

Then, for each author (or editor), we want to be able to create a file with the
correct name. To do that, we created the \hl{ufn:format\_full\_name()} function
which takes as parameter the full name of a person. The lastname is extracted
and put in the first position. Then, if the first name has multiple parts, they
are concatenated with '\_', and if '.' is present, it is then removed (e.g.
\emph{A. B. M. Shawkat Ali} becomes \emph{Ali.A\_B\_M\_Shawkat}).\\

Once the name is correctly formatted, we can use \hl{xsl:result-document} to
produce a document per author/editor.\\

\begin{lstlisting}
<xsl:result-document href="a-tree/{$first_letter}/{$formattedAuthor}.html">
    <html>
        <head>
            <title>Publication of <xsl:value-of select="$author"/></title>
        </head>
        <body>
            <xsl:call-template name="author_header"/>
            <xsl:call-template name="author_publication"/>
            <xsl:call-template name="coauthor_index"/>
        </body>
    </html>
</xsl:result-document>
\end{lstlisting}

\subsection{Publications}
For each person, we store the publications' years in a variable. Then, we iterate
over the publications sorted by year (descending). The variable previously stored let us
retrieve for a publication $p$, the ${p-1}^{th}$ publication's year in order to
know if we must display a year header.\\

\begin{lstlisting}
<xsl:template name="author_publication">
    <xsl:variable name="sortedYears" as="xs:integer*">
        <xsl:perform-sort select="/dblp/*[author = current-grouping-key()]/year |
            /dblp/*[editor = current-grouping-key()]/year">
            <xsl:sort select="." order="descending"/>
        </xsl:perform-sort>
    </xsl:variable>

    ...

    <xsl:for-each select="/dblp/*[author = current-grouping-key()
            or editor = current-grouping-key()]">
        <xsl:sort select="year" order="descending" />

        <xsl:variable name="prevPosition" select="position() - 1"/>
        <xsl:variable name="prevYear" select="$sortedYears[$prevPosition]"/>

        <xsl:if test="not(year = $prevYear)">
            <tr><th colspan="3" bgcolor="#FFFFCC">
                <xsl:value-of select="year"/></th>
            </tr>
        </xsl:if>

        ...
    </xsl:for-each>
\end{lstlisting}
\

\begin{framewarning}
We tested other methods to try to display year headers. The first one used
\hl{previous-sibling} in the \hl{for-each} loop, but it did not work, since the
elements from the loop are sorted by year, and the axis returns elements as if
they were not sorted. The second one used \hl{for-each-group}, grouped by year,
and probably was more readable, less verbose than what is used now. But it was
then hard to retrieve the publications' position, since the \hl{position()}
function returned the position inside the group, and not inside all the author's
publications. Moreover, since XSLT is a functional programming language, variables
are immutable, so implementing a counter was not an option (it can be done
recursively though, but it would not have helped us).
\end{framewarning}
\

Each publication is then handled inside the \emph{publication} named template.
The position is computed with the position in the for eahh loop and the total
number of publications ($number of publications - position() + 1$). A template
is applied on the publication itself in order to match one of the publication
type (e.g. book, article, masterthesis). Inside of these publication types
templates, templates are applied on sub-elements that can appear in these
publications. By applying template, optionnal elements are handled easily.
\newpage

\subsection{Co-authors index}
% TODO
\newpage


\begin{thebibliography}{5}
% TODO \bibitem{cite:REF} \emph{TITLE}, FROM, AUTHOR, \href{URL}{URL}
\bibitem{cite:w3c} \emph{XSL Transformations (XSLT) Version 2.0}, W3C,
	\href{http://www.w3.org/TR/xslt20/}{http://www.w3.org/TR/xslt20/}

\bibitem{cite:w3schools} \emph{XSLT Tutorial}, W3Schools,
	\href{http://www.w3schools.com/xsl/}{http://www.w3schools.com/xsl/}

\bibitem{cite:microhowto} \emph{Generate multiple output documents using XSLT}, microHOWTO,
	\href{http://www.microhowto.info/howto/generate_multiple_output_documents_using_xslt.html}
	{http://www.microhowto.info/ howto/generate\_multiple\_output\_documents\_using\_xslt.html}

\bibitem{cite:so} \emph{Differences between for-each and templates in xsl?}, Stack Overflow,
	\href{https://stackoverflow.com/questions/4460232/differences-between-for-each-and-templates-in-xsl}
	{https://stackoverflow.com/ questions/4460232/differences-between-for-each-and-templates-in-xsl}
\end{thebibliography}

\end{document}
